The \mbox{\hyperlink{class_d_s2482}{D\+S2482}} is an I2C to 1-\/wire interface chip. It comes in two versions, the D\+S2482-\/100 (1-\/port) and D\+S2482-\/800 (8-\/port). Using an interface chip is helpful because most D\+S18\+B20/1-\/wire libraries use timing sensitive code and may run for extended periods with interrupts disabled. This can cause the rest of your program to have poor performance.

The library fully supports both single-\/drop and multi-\/drop modes, allowing many D\+S18\+B20 sensors on a single 1-\/wire bus.

The \mbox{\hyperlink{class_d_s2482}{D\+S2482}} library is completely asynchronous, never blocking for more than the time to do an I2C read or write. Interrupts are never disabled.

Every call uses a C++11 lambda completion handler. The second part of this document has a bit of explanation of why and how it works.

The \mbox{\hyperlink{class_d_s2482}{D\+S2482}} also has an internal transistor to pull the 1-\/wire bus high during temperature conversion and flash writes. This allows the use of parasitic power mode, requiring only two wires for sensors (DQ and G\+ND), with no separate power line. The library supports this as well.

\subsection*{Common tasks}

\subsubsection*{Get temperature of one sensor (single-\/drop)}


\begin{DoxyCode}
#include "DS2482.h"

SerialLogHandler logHandler;

DS2482 ds(Wire, 3);

const unsigned long CHECK\_PERIOD = 30000;
unsigned long lastCheck = 5000 - CHECK\_PERIOD;

void setup() \{
    Serial.begin(9600);
    ds.setup();

    DS2482DeviceReset::run(ds, [](DS2482DeviceReset&, int status) \{
        Serial.printlnf("deviceReset=%d", status);
    \});

    Serial.println("setup complete");
\}



void loop() \{

    ds.loop();

    if (millis() - lastCheck >= CHECK\_PERIOD) \{
        lastCheck = millis();

        // For single-drop you can pass an empty address to get the temperature of the only
        // sensor on the 1-wire bus
        DS24821WireAddress addr;

        DS2482GetTemperatureCommand::run(ds, addr, [](DS2482GetTemperatureCommand&, int status, float
       tempC) \{
            if (status == DS2482Command::RESULT\_DONE) \{
                char buf[32];
                snprintf(buf, sizeof(buf), "%.4f", tempC);

                Serial.printlnf("temperature=%s deg C", buf);
                Particle.publish("temperature", buf, PRIVATE);
            \}
            else \{
                Serial.printlnf("DS2482GetTemperatureCommand failed status=%d", status);
            \}
        \});
    \}
\}
\end{DoxyCode}


\subsubsection*{Get temperatures of multiple sensors (multi-\/drop)}


\begin{DoxyCode}
#include "DS2482.h"

SerialLogHandler logHandler;

DS2482 ds(Wire, 3);

DS2482DeviceListStatic<10> deviceList;
const unsigned long CHECK\_PERIOD = 30000;
unsigned long lastCheck = 10000 - CHECK\_PERIOD;

void setup() \{
    Serial.begin(9600);
    ds.setup();

    DS2482DeviceReset::run(ds, [](DS2482DeviceReset&, int status) \{
        Serial.printlnf("deviceReset=%d", status);
        DS2482SearchBusCommand::run(ds, deviceList, [](DS2482SearchBusCommand &obj, int status) \{

            if (status != DS2482Command::RESULT\_DONE) \{
                Serial.printlnf("DS2482SearchBusCommand status=%d", status);
                return;
            \}

            Serial.printlnf("Found %u devices", deviceList.getDeviceCount());
        \});
    \});

    Serial.println("setup complete");
\}


void loop() \{

    ds.loop();

    if (millis() - lastCheck >= CHECK\_PERIOD) \{
        lastCheck = millis();

        if (deviceList.getDeviceCount() > 0) \{

            DS2482GetTemperatureForListCommand::run(ds, deviceList, [](DS2482GetTemperatureForListCommand&,
       int status, DS2482DeviceList &deviceList) \{
                if (status != DS2482Command::RESULT\_DONE) \{
                    Serial.printlnf("DS2482GetTemperatureForListCommand status=%d", status);
                    return;
                \}

                Serial.printlnf("got temperatures!");

                for(size\_t ii = 0; ii < deviceList.getDeviceCount(); ii++) \{
                    Serial.printlnf("%s valid=%d C=%f F=%f",
                            deviceList.getAddressByIndex(ii).toString().c\_str(),
                            deviceList.getDeviceByIndex(ii).getValid(),
                            deviceList.getDeviceByIndex(ii).getTemperatureC(),
                            deviceList.getDeviceByIndex(ii).getTemperatureF());
                \}

            \});
        \}
        else \{
            Serial.printlnf("no devices found");
        \}
    \}
\}
\end{DoxyCode}


\subsubsection*{Multi-\/drop with parasitic power}

Just as the previous example, except within loop the call to D2482\+Get\+Temperature\+List\+Command\+::run has an extra optional fluent parameter, {\ttfamily .with\+Parasitic\+Power()}.


\begin{DoxyCode}
DS2482GetTemperatureForListCommand::run(ds, deviceList, [](DS2482GetTemperatureForListCommand&, int status,
       DS2482DeviceList &deviceList) \{
    if (status != DS2482Command::RESULT\_DONE) \{
        Serial.printlnf("DS2482GetTemperatureForListCommand status=%d", status);
        return;
    \}

    Serial.printlnf("got temperatures!");

    for(size\_t ii = 0; ii < deviceList.getDeviceCount(); ii++) \{
        Serial.printlnf("%s valid=%d C=%f F=%f",
                deviceList.getAddressByIndex(ii).toString().c\_str(),
                deviceList.getDeviceByIndex(ii).getValid(),
                deviceList.getDeviceByIndex(ii).getTemperatureC(),
                deviceList.getDeviceByIndex(ii).getTemperatureF());
    \}

\}).withParasiticPower();
\end{DoxyCode}


\subsubsection*{Multi-\/drop with J\+S\+ON publish}

This example is like the previous, except it publishes multiple sensor values via a single Particle.\+publish in J\+S\+ON format (up to 10 D\+S18\+B20s supported).


\begin{DoxyCode}
#include "DS2482.h"
#include "JsonParserGeneratorRK.h"

SerialLogHandler logHandler;

DS2482 ds(Wire, 3);

DS2482DeviceListStatic<10> deviceList;
JsonWriterStatic<256> jsonWriter;

const unsigned long CHECK\_PERIOD = 30000;
unsigned long lastCheck = 10000 - CHECK\_PERIOD;

void setup() \{
    Serial.begin(9600);
    ds.setup();

    DS2482DeviceReset::run(ds, [](DS2482DeviceReset&, int status) \{
        Serial.printlnf("deviceReset=%d", status);
        DS2482SearchBusCommand::run(ds, deviceList, [](DS2482SearchBusCommand &obj, int status) \{

            if (status != DS2482Command::RESULT\_DONE) \{
                Serial.printlnf("DS2482SearchBusCommand status=%d", status);
                return;
            \}

            Serial.printlnf("Found %u devices", deviceList.getDeviceCount());
        \});
    \});

    Serial.println("setup complete");
\}


void loop() \{

    ds.loop();

    if (millis() - lastCheck >= CHECK\_PERIOD) \{
        lastCheck = millis();

        if (deviceList.getDeviceCount() > 0) \{

            DS2482GetTemperatureForListCommand::run(ds, deviceList, [](DS2482GetTemperatureForListCommand&,
       int status, DS2482DeviceList &deviceList) \{
                if (status != DS2482Command::RESULT\_DONE) \{
                    Serial.printlnf("DS2482GetTemperatureForListCommand status=%d", status);
                    return;
                \}

                Serial.printlnf("got temperatures!");

                // Initialize the JsonWriter object and sets it to send 2 decimal places
                jsonWriter.init();
                jsonWriter.setFloatPlaces(2);

                // startObject is for the outer object and must be balanced with finishObjectOrArray.
                jsonWriter.startObject();

                // Write the actual temperatures
                for(size\_t ii = 0; ii < deviceList.getDeviceCount(); ii++) \{
                    if (deviceList.getDeviceByIndex(ii).getValid()) \{
                        // This creates a key of the form "t0" for the first, "t1" for the second, ...
                        char key[4];
                        key[0] = 't';
                        key[1] = '0' + ii;
                        key[2] = 0;

                        // Inserts the float as a key value pair
                        jsonWriter.insertKeyValue(key, deviceList.getDeviceByIndex(ii).getTemperatureC());
                    \}
                \}

                jsonWriter.finishObjectOrArray();

                Serial.println(jsonWriter.getBuffer());
                Particle.publish("temperature", jsonWriter.getBuffer(), PRIVATE);

                // Example output:
                // \{"t0":23.56,"t1":23.12\}
            \});
        \}
        else \{
            Serial.printlnf("no devices found");
        \}
    \}
\}
\end{DoxyCode}


Example serial output\+:


\begin{DoxyCode}
got temperatures!
\{"t0":23.56,"t1":23.12\}
\end{DoxyCode}


Because the 1-\/wire bus search is deterministic, it will always returns the sensors in the same order, sorted by 1-\/wire device address (increasing).

\subsection*{About lambdas, fluent-\/style and more}

The \mbox{\hyperlink{class_d_s2482}{D\+S2482}} library uses a C++11 style of coding that is very powerful, but will probably be foreign to you if you learned old-\/school C and C++. It\textquotesingle{}s actually more like the way Javascript/node.\+js is programmed.

One of the main advantages of the \mbox{\hyperlink{class_d_s2482}{D\+S2482}} library is that it\textquotesingle{}s completely asynchronous. No call blocks for more than the time it takes to make an I2C transfer. This is important because getting a temperature sensor reading can take 750 milliseconds, and most D\+S18\+B20 software libraries block during conversion.

\subsubsection*{Synchronous way}

Using the \href{https://build.particle.io/libs/DS18B20/0.1.7/tab/DS18B20.cpp}{\tt D\+S18\+B20 library}, you make a call like this\+:


\begin{DoxyCode}
float tempC = ds.getTemperature(addr);
\end{DoxyCode}


The main loop thread is blocked during this call for 750 milliseconds, which may affect performance elsewhere in your code.

\subsubsection*{State machines}

One common way of implementing asynchronous code is state machines. This is a bit of pseudo-\/code of what a state machine version might look like.


\begin{DoxyCode}
void loop() \{
    switch(state) \{
    case GET\_TEMPERATURE\_STATE:
        ds.startTemperature();
        state = TEMPERATURE\_WAIT\_STATE;
        break;

    case TEMPERATURE\_WAIT\_STATE:
        if (ds.isTemperatureDone()) \{
            float temp = ds.readTemperature();
            state = ANOTHER\_STATE;
        \}
        break;
    \}
\}
\end{DoxyCode}


State machines are really powerful, and actually how the \mbox{\hyperlink{class_d_s2482}{D\+S2482}} library is implemented internally, but can get a bit unwieldy.

\subsubsection*{Callbacks}

Callback functions are another common way of handling asynchronous code, like in this pseudo-\/code example\+:


\begin{DoxyCode}
void temperatureCallback(float temp) \{
    Serial.printlnf("got temperature %f", temp);
\}

void loop() \{
    if (millis() - lastCheck >= CHECK\_PERIOD) \{
        lastCheck = millis();
        ds.getTemperature(temperatureCallback);
    \}
\}
\end{DoxyCode}


The problem with callbacks is that it\textquotesingle{}s a pain to pass state data to the callback, and it requires some effort to make a callback handler a class member function.

\subsubsection*{The Lambda}

The lambda solves some of the problems with callbacks. Here\textquotesingle{}s an example of doing an asynchronous device reset call\+:


\begin{DoxyCode}
void setup() \{
    Serial.begin(9600);
    ds.setup();

    DS2482DeviceReset::run(ds, [](DS2482DeviceReset&, int status) \{
        Serial.printlnf("deviceReset=%d", status);
    \});

    Serial.println("setup complete");
\}
\end{DoxyCode}


The syntax is a little weird, and I\textquotesingle{}ll get to that a moment, however the important thing is that this block of code, the lambda, is executed later.


\begin{DoxyCode}
[](DS2482DeviceReset&, int status) \{
    Serial.printlnf("deviceReset=%d", status);
\}
\end{DoxyCode}


The


\begin{DoxyCode}
setup complete
\end{DoxyCode}


message appears next.

When the asynchronous device reset completes, the block runs and


\begin{DoxyCode}
deviceReset=1
\end{DoxyCode}


will likely be printed to the serial console. Note, however, that the lines after the block, like printing setup complete! won\textquotesingle{}t happen a second time.

A declaration of a lambda function looks like


\begin{DoxyCode}
[](DS2482DeviceReset&, int status) \{
    // Code goes here
\}
\end{DoxyCode}


The {\ttfamily \mbox{[}\mbox{]}} part is the capture, which we\textquotesingle{}ll discuss in a moment, and {\ttfamily (\mbox{\hyperlink{class_d_s2482_device_reset}{D\+S2482\+Device\+Reset}}\&, int status)} is a function parameter declaration. Basically the callback is a function that takes two parameters, a {\ttfamily \mbox{\hyperlink{class_d_s2482_device_reset}{D\+S2482\+Device\+Reset}}\&} object (not used here) and an {\ttfamily int status} (for status).

\subsubsection*{Nesting calls}

This example demonstrates two handy things\+:


\begin{DoxyItemize}
\item Making your lambda be a class member
\item Nesting lambdas for sequential operations
\end{DoxyItemize}


\begin{DoxyCode}
void TestClass::check() \{
    DS2482SearchBusCommand::run(ds, deviceList, [this](DS2482SearchBusCommand &obj, int status) \{

        if (status != DS2482Command::RESULT\_DONE) \{
            Serial.printlnf("DS2482SearchBusCommand status=%d", status);
            return;
        \}

        if (deviceList.getDeviceCount() == 0) \{
            Serial.println("no devices");
            return;
        \}

        DS2482GetTemperatureForListCommand::run(ds, obj.getDeviceList(),
       [this](DS2482GetTemperatureForListCommand&, int status, DS2482DeviceList &deviceList) \{
            if (status != DS2482Command::RESULT\_DONE) \{
                Serial.printlnf("DS2482GetTemperatureForListCommand status=%d", status);
                return;
            \}

            Serial.printlnf("got temperatures!");

            for(size\_t ii = 0; ii < deviceList.getDeviceCount(); ii++) \{
                Serial.printlnf("%s valid=%d C=%f F=%f",
                        deviceList.getAddressByIndex(ii).toString().c\_str(),
                        deviceList.getDeviceByIndex(ii).getValid(),
                        deviceList.getDeviceByIndex(ii).getTemperatureC(),
                        deviceList.getDeviceByIndex(ii).getTemperatureF());
            \}

        \});
    \});
\}
\end{DoxyCode}


You\textquotesingle{}ll notice the slightly different syntax in the capture, the part in the square brackets\+:


\begin{DoxyCode}
DS2482SearchBusCommand::run(ds, deviceList, [this](DS2482SearchBusCommand &obj, int status) \{
\end{DoxyCode}


Instead of just being {\ttfamily \mbox{[}\mbox{]}} it\textquotesingle{}s {\ttfamily \mbox{[}this\mbox{]}}. That means that {\ttfamily this}, your class instance, is captured, and available inside the lambda. It essentially makes the inner function a class member, available to access class member functions and variables.

You can capture multiple variables, separated by commas. You can capture function parameters and local variables, for example.

The other thing is that you can nest these, so each new indent in occurs at a later time. If you don\textquotesingle{}t like that style, however, you can just split your code into separate member functions like this\+:


\begin{DoxyCode}
void TestClass::searchBus() \{

    DS2482SearchBusCommand::run(ds, deviceList, [this](DS2482SearchBusCommand &obj, int status) \{

        if (status != DS2482Command::RESULT\_DONE) \{
            Serial.printlnf("DS2482SearchBusCommand status=%d", status);
            return;
        \}

        if (deviceList.getDeviceCount() == 0) \{
            Serial.println("no devices");
            return;
        \}

        getTemperatures();
    \});
\}

void TestClass::getTemperatures() \{

    DS2482GetTemperatureForListCommand::run(ds, deviceList, [this](DS2482GetTemperatureForListCommand&, int
       status, DS2482DeviceList &deviceList) \{
        if (status != DS2482Command::RESULT\_DONE) \{
            Serial.printlnf("DS2482GetTemperatureForListCommand status=%d", status);
            return;
        \}

        Serial.printlnf("got temperatures!");

        for(size\_t ii = 0; ii < deviceList.getDeviceCount(); ii++) \{
            Serial.printlnf("%s valid=%d C=%f F=%f",
                    deviceList.getAddressByIndex(ii).toString().c\_str(),
                    deviceList.getDeviceByIndex(ii).getValid(),
                    deviceList.getDeviceByIndex(ii).getTemperatureC(),
                    deviceList.getDeviceByIndex(ii).getTemperatureF());
        \}
    \});

\}
\end{DoxyCode}


\subsubsection*{Optional parameters}

Within the library, optional parameters are passed fluent-\/style instead of using C++ optional parameters. The fluent-\/style parameters are easier to identify and don\textquotesingle{}t depend on order. For example\+:


\begin{DoxyCode}
DS2482GetTemperatureCommand::run(ds, addr, [this,completion](DS2482GetTemperatureCommand&, int status,
       float t) \{
        if (status != DS2482Command::RESULT\_DONE) \{
            Log.error("FAILURE DS2482GetTemperatureCommand failed %d bus %s", status, name.c\_str());
            completion();
            return;
        \}

        testsComplete(completion);
    \}).withParasiticPower(isParasiticPowered).withMaxRetries(1);
\end{DoxyCode}


Of note\+:


\begin{DoxyItemize}
\item The optional parameters go after the {\ttfamily \})} that closes the run command.
\item You can chain as many or as few as you want in any order.
\item The names all begin with {\ttfamily with}.
\end{DoxyItemize}

In this example, it uses parasitic power or not based on the member variable {\ttfamily is\+Parasitic\+Powered} and limits the number of retries to 1.

\subsubsection*{Run methods}

All of the asynchronous functions are implemented as separate classes with a static run method. For example {\ttfamily \mbox{\hyperlink{class_d_s2482_get_temperature_command_a91c9ee5048047d209e3dd1effa1ba179}{D\+S2482\+Get\+Temperature\+Command\+::run}}}.

This is done because the objects need to be allocated on the heap (using new), not stack allocated. Since they continue after your function returns, stack allocated classes wouldn\textquotesingle{}t work because they would go away when the enclosing function returns.

The run methods take care of allocated the objects and enqueueing them for execution so they make sure the objects gets delete later, as well.

\subsection*{Library documentation}

The full library documentation can be found here. It\textquotesingle{}s automatically generated from the .h file, so you can read the comments there as well.

\subsection*{Hardware examples}